\documentclass[UTF-8,a4paper]{ctexart}
\usepackage{amsmath,amsfonts}

%title
\begin{document}
\title{Flipping game under the perspective of Linear Algebra\\寻找变化中的不变量}
\author{\kaishu 俞奕成}
\date{\kaishu \today}
\maketitle
%contents
\setcounter{tocdepth}{2}
\tableofcontents
\newpage

\part{Introduction}
\section{开端:2阶翻棋游戏的具体描述}
问题的开始是一个简单的二阶翻棋游戏:\par
初始状况:\par
\kaishu 在\(2 \times 2\)的棋盘上,随机放置着一些棋子,这些棋子一面是黑色,一面是白色,向上面的颜色完全随机 \par
\songti 允许对这些棋子进行如下操作:\par
\kaishu 每次选中一列或者一行棋子,将这一列或者一行棋子全部翻转。\par
\songti 游戏要求的最终结果:\par
\kaishu 如果有可能,在进行若干次操作之后,棋盘上的棋子变为全部黑色向上或全部白色向上。\par
\songti 游戏的解法:\par
\kaishu 一系列允许的操作,使得初始情况经过这些操作后变为最终结果 \par
\songti 注意:\par
\kaishu 并不是所有棋盘都有可能达到最终要求,部分初始状况在操作后是不可能达到最终要求的
\section{问题的提出}
\songti 对于这个翻棋游戏,不难提出如下问题:
\kaishu
\begin{enumerate}
    \item 有没有方法,在给出任意一种棋盘的初始状况后,快速的判断出该棋盘是否有可能达到要求的结果(向上面全黑或全白)?
    \item 有没有方法,对于给出的任意一种棋盘的初始状况,如果有可能达到要求结果,都能找出一种能该游戏的解法?
\end{enumerate}

\section{过程和问题的抽象化}
\songti
棋盘,黑棋子,白棋子显然不利于我们以线性代数的方式来思考问题。所以,我们应该借助线性代数的概念将问题抽象化: 
\kaishu
\\定义棋盘为一个\(2 \times 2\)的矩阵:
\[\begin{pmatrix}
    a_{11}&a_{12}\\
    a_{21}&a_{22}\\
\end{pmatrix}\]
\quad
定义:白棋子为\(1\),黑棋子为\(-1\)
\\允许的操作也可以对应到矩阵的变换:将某一行(列)的元素全部乘以\(1\)或者\(-1\)
\\不难发现这个操作就是矩阵的一种初等行(列)变换
\\游戏的要求——最终变为全黑或者全白,也可以对应到元素全为\(-1\)或者元素全为\(1\)的矩阵,即\\
\begin{center}
\(\begin{pmatrix}
    -1&-1\\
    -1&-1\\
\end{pmatrix}\)
\quad
(全黑)
\(\begin{pmatrix}
    1&1\\
    1&1\\
\end{pmatrix}\)
\quad
(全白)
\end{center}


\section{具体问题的扩展}
只要将翻棋游戏的定义中\(2 \times 2\)棋盘改为\(n \times n\)棋盘,而其他定义不变,翻棋游戏的规则就可以很容易地扩展到\(n \times n\)的棋盘上。 \\
两个问题在\(n \times n\)棋盘的情况下仍然存在并且可以讨论。
\section{抽象定义的扩展}
只需将\(2 \times 2\)的矩阵扩展为\(n \times n\)的矩阵:
\[\begin{pmatrix}
    a_{11}&a_{12}&\dots&a_{1n}
    \\a_{21}&a_{22}&\dots&a_{2n}
    \\ \vdots &\vdots &\vdots &\vdots
    \\a_{n1}&a_{n2}&\dots&a_{nn}
\end{pmatrix}\]
游戏要求的最终结果也进行对应扩展:
\[\begin{pmatrix}
    1&1&\dots&1
    \\1&1&\dots&1
    \\ \vdots &\vdots &\vdots &\vdots
    \\1&1&\dots&1
\end{pmatrix}\]
\\其余定义不变,即可将抽象定义扩展到\(n \times n\)

\part{初始简单情况的规律与分析}
\songti 先从
\part{使用线性代数的方式分析该问题}
\part{一般化情况的分析与思考}
\part{该类问题一般解法的实现与分析}
\end{document}