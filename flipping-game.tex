\documentclass[UTF-8,a4paper]{ctexart}
\usepackage{amsmath,amsfonts}
\newtheorem{definition}[subsubsection]{定义}
\newtheorem{theorem}[subsubsection]{定理}
\newtheorem{property}[subsubsection]{性质}

%title
\begin{document}
\title{Flipping game under the perspective of Linear Algebra\\寻找变化中的不变量}
\author{\kaishu 俞奕成}
\date{\kaishu \today}
\maketitle
%contents
\setcounter{tocdepth}{2}
\tableofcontents
\newpage

\part{Introduction}
\section{开端:2阶翻棋游戏的具体描述}
问题的开始是一个简单的二阶翻棋游戏:\par
初始状况:\par
\kaishu 在\(2 \times 2\)的棋盘上,随机放置着一些棋子,这些棋子一面是黑色,一面是白色,向上面的颜色完全随机 \par
\songti 允许对这些棋子进行如下操作:\par
\kaishu 每次选中一列或者一行棋子,将这一列或者一行棋子全部翻转。\par
\songti 游戏要求的最终结果:\par
\kaishu 如果有可能,在进行若干次操作之后,棋盘上的棋子变为全部黑色向上或全部白色向上。\par
\songti 游戏的解法:\par
\kaishu 一系列允许的操作,使得初始情况经过这些操作后变为最终结果 \par
\songti 注意:\par
\kaishu 并不是所有棋盘都有可能达到最终要求,部分初始状况在操作后是不可能达到最终要求的
\section{问题的提出}
\songti 对于这个翻棋游戏,不难提出如下问题:
\kaishu
\begin{enumerate}
    \item 有没有方法,在给出任意一种棋盘的初始状况后,快速的判断出该棋盘是否有可能达到要求的结果(向上面全黑或全白)?
    \item 有没有方法,对于给出的任意一种棋盘的初始状况,如果有可能达到要求结果,都能找出一种能该游戏的解法?
\end{enumerate}

\section{过程和问题的抽象化}
\songti
棋盘,黑棋子,白棋子显然不利于我们以线性代数的方式来思考问题。所以,我们应该借助线性代数的概念将问题抽象化: 
\kaishu
\\定义棋盘为一个\(2 \times 2\)的矩阵:
\[\begin{pmatrix}
    a_{11}&a_{12}\\
    a_{21}&a_{22}\\
\end{pmatrix}\]
\quad
定义:白棋子为\(1\),黑棋子为\(-1\)
\\由此可知,矩阵的初始情况的所有元素都是随机的\(1\)和\(-1\)
\\例如:
\[\begin{pmatrix}
    1&-1\\
    -1&1\\
\end{pmatrix}\]
\quad
\\允许的操作也可以对应到矩阵的变换:将某一行(列)的元素全部乘以\(1\)或者\(-1\)
\\不难发现这个操作就是矩阵的一种初等行(列)变换
\\游戏的要求——最终变为全黑或者全白,也可以对应到元素全为\(-1\)或者元素全为\(1\)的矩阵,即\\
\begin{center}
\(\begin{pmatrix}
    -1&-1\\
    -1&-1\\
\end{pmatrix}\)
\quad
(全黑)
\(\begin{pmatrix}
    1&1\\
    1&1\\
\end{pmatrix}\)
\quad
(全白)
\end{center}


\section{具体问题的扩展}
只要将翻棋游戏的定义中\(2 \times 2\)棋盘改为\(n \times n\)棋盘,而其他定义不变,翻棋游戏的规则就可以很容易地扩展到\(n \times n\)的棋盘上。 \\
两个问题在\(n \times n\)棋盘的情况下仍然存在并且可以讨论。
\section{抽象定义的扩展}
只需将\(2 \times 2\)的矩阵扩展为\(n \times n\)的矩阵:
\[\begin{pmatrix}
    a_{11}&a_{12}&\dots&a_{1n}
    \\a_{21}&a_{22}&\dots&a_{2n}
    \\ \vdots &\vdots &\vdots &\vdots
    \\a_{n1}&a_{n2}&\dots&a_{nn}
\end{pmatrix}\]
游戏要求的最终结果也进行对应扩展:
\[\begin{pmatrix}
    1&1&\dots&1
    \\1&1&\dots&1
    \\ \vdots &\vdots &\vdots &\vdots
    \\1&1&\dots&1
\end{pmatrix}
\quad
\]
\[\begin{pmatrix}
    -1&-1&\dots&-1
    \\-1&-1&\dots&-1
    \\ \vdots &\vdots &\vdots &\vdots
    \\-1&-1&\dots&-1
\end{pmatrix}
\quad
\]
\\其余定义不变,即可将抽象定义扩展到\(n \times n\)

\part{初始简单情况的规律与分析}
\songti 先从\(2 \times 2\)的情况开始分析,由于这种情况很简单,初始情况只有\(2^4=16\)种,我们不妨把每种情况都枚举一下,并进行分类:\\
\kaishu
分类为可以转化的:
\(\begin{pmatrix}
    1&-1\\
    -1&1\\
\end{pmatrix}\)
\quad
\(\begin{pmatrix}
    -1&1\\
    1&-1\\
\end{pmatrix}\)
\quad
\dots
\\和不能转化的:
\(\begin{pmatrix}
    1&1\\
    -1&1\\
\end{pmatrix}\)
\quad
\(\begin{pmatrix}
    -1&1\\
    1&1\\
\end{pmatrix}\)
\quad
\songti
\\能转化的矩阵自然可以直接给出方法,而不能给出的矩阵需要进行证明,下面给出一种简易的证明:
\kaishu
\\每一次操作改变的元素都是偶数个,故不会改变矩阵内所有元素的乘积,初始状况所有元素乘积是\(-1\)的矩阵无论经过多少次变化,乘积仍然是\(-1\),而不难发现最后全黑或者全白的话,所有元素的乘积是1,故这些矩阵不可能变为全黑或者全白
\songti
\\我们可以发现以上的证明过程已经给出了一种判断\(2 \times 2\)矩阵是否可以达到最终结果的特征:即
\kaishu
\begin{enumerate}
    \item 若所有元素乘积为1,则可以转化
    \item 若所有元素乘积为\(-1\),则不能转化
\end{enumerate}
\songti
对这些矩阵的特点进行分析,不难发现另外一些可以用于判断的特征:
\kaishu
\\设初始矩阵为A,发现
\begin{enumerate}
    \item 能够转化的矩阵均有行列式为0(\(det(A)=0\))
    \item 而不能转化的矩阵均有行列式不为0(\(det(A)\neq 0\))
\end{enumerate}
\songti
由于\(2 \times 2\)的情况全部都枚举出来了,所以上面那些检验标准的正确性都可以直接验证。
\part{使用线性代数的方式分析该问题}
\section{很多2阶标准并不适用于高阶情况}
接下来我们要考虑\(n \times n\)的情况了。\\
首先可以尝试将前面提出的2阶情况下的两种判断标准直接扩展到高阶。\\
但是可以发现直接使用元素乘积判断显然是不可行的,因为一次操作会改变奇数个元素\\
而直接使用行列式判断也不可行,因为可以举出反例:\\
\kaishu
例如三阶行列式
\(\begin{pmatrix}
    -1&1&1\\
    1&-1&-1\\
    -1&1&1\\
\end{pmatrix}\)
\quad
并不能化为需要的最终结果,但是它的行列式却是0\\
\songti
可见前面的两个适用于2阶的标准并不适用于高阶的一般情况。
\section{引入更多线性代数的概念和思考方式}
前面的第二种采用行列式的思考方式虽然不能直接扩展,但尝试将它推广很容易让我们联想到线性代数中的一个概念——矩阵的秩。
这样做有两个好处:
\kaishu
\begin{enumerate}
    \item 我们允许的操作是初等行(列)变换,这个过程中矩阵的秩具有不变性,对分析有利
    \item 2阶情况下,对非零矩阵(这边的初始情况当然不可能是零矩阵)而言,行列式等于0与秩为1是等价的,猜想这可能是更好的推广方式。
\end{enumerate}
\songti
\part{一般化情况的分析与思考}
\section{判断标准:矩阵的秩为1}
\subsection{推断,猜想和尝试}
前面已经提到了猜想,秩为1是很有可能是一个正确的判断方法。
我们可以列举几个高阶情况,发现这个标准都是正确的。
由于变换(操作)过程中矩阵的秩不变,我们希望寻找变化中的不变,这样易于解决问题。
\\因此对于不变的最终结果——全黑或者全白矩阵进行分析,我们发现这两个矩阵的秩可能等于1。
\\于是,下面给出简易证明:
\subsection{最终结果矩阵秩为1的证明}
\kaishu
由定义可知,在表示全黑(全白)棋盘的\(n \times n\)矩阵\(\mathbf{A}=(a_{ij})_{n \times n}\)中,
\\有\(\forall i,j , a_{ij}=1(or -1) \)
\\从中任意取出一个元素\(a_{ij}\),即为该矩阵的一阶子式,显然\(a_{ij} \neq 0\),因此存在不为零的一阶子式
\\另一方面,矩阵\(\mathbf{A}\)中\(\forall\)二阶子式\(\begin{vmatrix}
    a_{ij}&a_{in}
    \\a_{mj}&a_{mn}
\end{vmatrix}\),有
\[\begin{vmatrix}
    a_{ij}&a_{in}
    \\a_{mj}&a_{mn}
\end{vmatrix}=a_{ij}a_{mn}-a_{in}a_{mj}=1-1=0\]
故由矩阵秩的定义,可知全黑或全白矩阵\(r(\mathbf{A})=1\)


\subsection{判断标准必要性证明}
\songti 在已经证明最终结果矩阵秩为1的基础上,容易证明判断标准的必要性:
\kaishu
\\翻棋操作是一种初等变换,而初等变换不会改变矩阵的秩
\\ \(\Rightarrow\)任意次允许的操作都不会改变矩阵的秩
\\\(\Rightarrow\)如果给定的一个初始情况棋盘对应的\(n \times n\)矩阵\(\mathbf{M}\)经过若干次允许的操作能够化为全黑或全白矩阵\(\mathbf{A}\),则必有\(r(\mathbf{M})=1\)
\\即\quad一个棋盘能化为全黑或全白\(\Rightarrow\)对应矩阵的秩\(r(\mathbf{M})=1\)

\subsection{判断标准充分性证明}
\songti
上述证明过程尚不足以说明秩为1这个判断标准的充分性。我们先明确充分性问题:
\kaishu
\\分析:
\\ \(\forall\)一个\(r(\mathbf{M})=1\)的\(n \times n\)初始情况矩阵\(\mathbf{M}\)
\\证明:它可以经过若干次\(-1c_i\)和\(-1r_i\)初等变换,化为全1或全-1矩阵
\\ \songti 经过线性代数的学习,我们知道初等变换\(-1r_i\)等价于左乘初等矩阵\(E(i(-1))\),初等变换\(-1c_i\)等价于右乘初等矩阵\(E(i(-1))\)
\\于是,可以进一步从矩阵的角度来描述这个问题:
\kaishu
\\存在\(n \times n\)矩阵\(\mathbf{P}, \mathbf{Q},s.t. \mathbf{P}\mathbf{M}\mathbf{Q}=\mathbf{A}\)
其中\[\mathbf{P},\mathbf{Q}=\begin{pmatrix}
    a_{11}
    \\&a_{22}
    \\&& \ddots
    \\ &&& a_{nn}
\end{pmatrix} \quad
a_{ii}= 1\,or\,-1 \quad
i = 1,2,\dots,n
\]
\(\mathbf{M}\)为棋盘初始情况对应的\(n \times n\)矩阵
\\\(\mathbf{A}\)为全黑或全白的棋盘最终结果对应的\(n \times n\)矩阵
\songti
\\然而,接下来,我们发现这个问题似乎难以从矩阵角度继续分析或者得出更多的结论了。
\\其实上面这个命题一共对应了有限的\(2^{2n}\)种情况。如果在有计算机辅助的情况下进行枚举,最终可以找出可行情况,将可行情况中的\(\mathbf{P},\mathbf{Q}\)还原成对应的行变换和列变换,再还原为允许的操作,就找出解法了。
\\然而这种的时间复杂度为\(\mathbf{O}(2^n)\),这是很高的一个时间复杂度,而且它过于暴力,很不数学,至少很不线性代数。
\\所以,我们考虑换个角度看问题,找到更多变化中的不变,引入了线性代数中和矩阵十分相关的概念——向量组。
\kaishu
\\线性代数中,有关向量组和矩阵的秩的关系,有如下定理:
\begin{theorem}
    \label{vectorR}
    设\(\mathbf{A}\)是\(m \times n\)矩阵,\\则\(\mathbf{A}\)的列向量组\(\alpha_1,\alpha_2,\dots,\alpha_n\)的秩等于矩阵\(\mathbf{A}\)的秩,
    \\ \(\mathbf{A}\)的行向量组的秩也等于\(\mathbf{A}\)的秩
\end{theorem}


\section{推论:更直观的判断标准}
\songti
但是,当矩阵的阶数很高的时候,想要快速的求出一个没有规律的矩阵的秩,或者得知它的秩是否是1是相当复杂而耗时的。


\section{找出一般化解答方法}

\part{该类问题一般解法的实现与分析}
\section{回顾解决翻棋游戏的过程:对线性代数中不变量的思考}
\section{关于其他学科及生活中不变量的思考}
\end{document}